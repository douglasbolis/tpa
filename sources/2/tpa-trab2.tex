\documentclass[a4paper,12pt]{scrartcl}

\usepackage[utf8]{inputenc}
\usepackage[brazil]{babel}
\usepackage[margin=25mm,bottom=30mm]{geometry}
\usepackage{hyperref}
\usepackage{amsmath,amssymb}
\usepackage{algorithm}
\usepackage{algpseudocode}
\usepackage{minted}
\usepackage{tabto}
\usepackage{siunitx}


\newcommand{\algorithmautorefname}{Algoritmo}


\title{Técnicas Avançadas de Programação}
\subtitle{Trabalo 2 -- Ordenação e Estatísticas de Ordem\\ Ifes -- Campus Serra}
\author{Prof. Dr. Jefferson O. Andrade}
\date{2018-2}


\begin{document}

\maketitle

\tableofcontents

\section{Introdução}

A ordenação é um dos problemas algorítmicos mais fundamentais da ciência da
computação. Já foi afirmado que até 25\% de todos os ciclos de CPU são usados
para ordenação, o que fornece um grande incentivo para estudos adicionais e
otimização sobre ordenação. Além disso, há muitas outras tarefas (pesquisar,
calcular a mediana, encontrar a moda, remover duplicatas etc.) que podem ser
implementadas com muito mais eficiência depois que os dados são ordenados. O
estudo da ordenação é particularmente interessante, visto que há muitas
abordagens diferentes (dividir para conquistar, particionamento,
\emph{pigeonhole}, etc.) que conduzem a algoritmos de ordenação úteis, e essas
técnicas podem muitas vezes ser reutilizadas para outras tarefas algorítmicos. A
grande variedade de algoritmos nos dá muita riqueza para explorar, especialmente
quando consideramos os \emph{tradeoffs} oferecidos em termos de eficiência, mix
de operações, complexidade de código, melhor/pior caso de entrada, e assim por
diante. Este trabalho foca na implementação e análise de alguns dos algoritmos
de ordenação.


\section{Enunciado}

Este trabalho consiste de duas etapas:

\begin{enumerate}
\item Implementação de um conjunto de algoritmos de ordenação.
\item Análise do desempenho dos algoritmos implementados.
\end{enumerate}

A descrição destas duas etapas será vista com mais detalhes abaixo.


\subsection{Implementação}
\label{sec:implementacao}

Esta etapa consiste na implementação dos algoritmos indicados na
\autoref{tab:algorithms}. Estes algoritmos deverão ser implementados em uma
linguagem de programação à escolha do aluno. Todo o trabalho deverá ser
implementado em uma mesma linguagem.

\begin{table}[t]
  \caption{Algorimos de ordenação selecionados para o trabalho.}
  \label{tab:algorithms}
  \begin{center}
    \begin{tabular}{|l|l|}
      \hline
      \textbf{Algoritmo} & \textbf{Identificador} \\
      \hline\hline
      Selection sort     & \texttt{selectsort}    \\
      \hline
      Insertion sort     & \texttt{insertsort}    \\
      \hline
      Merge Sort         & \texttt{mergesort}     \\
      \hline
      Quicksort          & \texttt{quicksort}     \\
      \hline
      Heapsort           & \texttt{heapsort}      \\
      \hline
    \end{tabular}
  \end{center}
\end{table}

Os algoritmos de ordenação devem ser implementados de tal modo que eles
funcionem para ordenação de \textbf{registros} em um vetor de acordo com uma
\textbf{chave de ordenação}. Uma forma relativamente simples de se conseguir
esse efeito é pasar uma função/objeto de comparação como argumento para o
procedimento de ordenação, e utilizar esta função/objeto para realizar as
comparações, ao invés de usar os operadores relacionais diretamente. Esta
abordagem é seguida, por exemplo, pela função \texttt{qsort}, da linguagem
C\footnote{%
  \mintinline{c}{qsort(void *ptr, size_t count, size_t size, int (*comp)(void *,
    void *))}}, e pelo método \texttt{sort}, da classe \texttt{Collections} em
Java\footnote{\mintinline{java}{sort(List<T> list, Comparator<? super T> c)}}.

O tempo de execução absoluto não será importante, portanto, em princípio, mesmo
uma linguagem ``lenta'' com Python pode ser utilizada sem prejuízo do trabalho.
A única imposição com relação à linguagem utilizada é que o ambiente de
desenvolvimento da linguagem (compilação, execução, etc.) deve ser livre,
gratuito e disponível para instalação no sistema Operacional GNU Linux nas
distribuição Ubuntu através do comando \texttt{apt}.

Os dados dos registros serão lidos de um arquivo texto em formato CSV
(\emph{Comma-separated
  Values})\footnote{\url{https://en.wikipedia.org/wiki/Comma-separated_values}}.
Depois de lidos do arquivo de entrada e ordenados, os dados ordenados devem ser
gravados em um arquivo de saída, também no formato CSV. A leitura e a gravação
dos arquivos CSV não é o objeto de avaliação deste trabalho, portante, o aluno
pode utilizar alguma biblioteca já pronta para estas atividades, ou utilizar
código já desenvolvido por terceiros, como o código disponível no sítio web
Rosetta
Code\footnote{\url{https://www.rosettacode.org/wiki/CSV_data_manipulation}}, por
exemplo.

Conforme citado na \autoref{sec:implementacao} os algoritmos devem receber uma
função ou objeto que faça a comparação. Para o teste do trabalho e a coleta dos
tempos de execução a ordenação deve ser feita sempre pelo campo
\emph{Identificador de Usuário} (uid).


\begin{algorithm}
  \caption{Template para programa de ordenação.}
  \label{alg:main}
  \begin{algorithmic}
\Procedure{main}{args}
    \State $\text{inputName}\gets\Call{parseInputFileName}{\text{args}}$
    \State $\text{outputName}\gets\Call{parseOutputFileName}{\text{args}}$
    \State $A\gets\Call{readCsv}{\text{inputName}}$
    \State $\text{initTime}\gets\Call{getCpuTime}{}$
    \State $\Call{xyzSort}{A}$
    \State $\text{finishTime}\gets\Call{getCpuTime}{}$
    \State \Call{writeCsv}{$A$, outputName}
    \State \Call{reportTime}{$A$, initTime, finishTime}
\EndProcedure
\end{algorithmic}
\end{algorithm}


O \autoref{alg:main} apresenta um modelo geral do que se espera do programa
gerado para execução dos algorimtos de ordenação. Note que o tempo é medido
apenas em relação à execução do algoritmo de ordenação. O último passo, i.e.,
relatar o tempo de execução do algoritmo, deve produzir uma única linha de saída
(para cada execução do algoritmos de ordenação) contendo as seguintes
informações, em ordem:

\begin{enumerate}
\item O identificador do algoritmo (veja \autoref{tab:algorithms}).
\item A quantidade de registros ordenados.
\item O tempo de execução do algoritmo de ordenação em milisegundos.
\end{enumerate}

Um exemplo de uso do programa é mostrado abaixo:\\

\begin{verbatim}
$$ ./isort -i data100.csv -o sorted100.csv
insertsort      100             0
\end{verbatim}

No exemplo acima, a linha que começa com os caracteres sifrão (\$\$) contém o
comando executado no terminal. Note que o nome \texttt{isort} é meramente uma
sugestão. O nome do arquivo de entrada foi indicado pela opção \texttt{-i} e o
nome do arquivo de saída foi indicado pela opção \texttt{-o}. Após a execução do
programa, foi gerada a saída contendo o identificador do algoritmo
(\texttt{insertsort}), o número de registros ordenados (100), e o tempo gasto
para ordenação em milisegundos (0).

O modelo apresentado no \autoref{alg:main} supõe que será criado um programa
para executar cada algoritmo de ordenação. Também é possível criar apenas um
programa principal que permita selecionar o algoritmo que será usado para
ordenação.

\begin{algorithm}
  \caption{Template para programa de ordenação com seleção do algoritmo.}
  \label{alg:main-opt}
  \begin{algorithmic}
\Procedure{main}{args}
    \State $\text{sort}\gets\Call{selectAlgorithm}{\text{args}}$
    \State $\text{inputName}\gets\Call{parseInputFileName}{\text{args}}$
    \State $\text{outputName}\gets\Call{parseOutputFileName}{\text{args}}$
    \State $A\gets\Call{readCsv}{\text{inputName}}$
    \State $\text{initTime}\gets\Call{getCpuTime}{}$
    \State $\Call{sort}{A}$
    \State $\text{finishTime}\gets\Call{getCpuTime}{}$
    \State \Call{writeCsv}{$A$, outputName}
    \State \Call{reportTime}{$A$, initTime, finishTime}
\EndProcedure
\end{algorithmic}
\end{algorithm}

O \autoref{alg:main-opt} apresenta uma versão modificada do modelo de programa
principal. Nesta versão, a função que fará a ordenação é selecionada de acordo
com o que for indicado nos argumentos de linha de comando. A função selecionada
é depois aplicada ao vetor de entrada. O restante do modelo é idêntico à versão
anterior. O exemplo abaixo mostra um exemplo/sugestão de uso do programa
principal implementado de acordo com o modelo do \autoref{alg:main-opt}.

\begin{verbatim}
$$ ./mysort --algorithm=sel -i data500.csv -o sorted500.csv
selectsort      500             13
\end{verbatim}


\subsection{Análise}

O processo de análise dos algoritmos implementados será feito por
experimentação. Serão fornecidos alguns arquivos de dados para que os algoritmos
sejam testados e sejam coletados os dados dos tempos de execução. Esses arquivos
de dados possuem números crescentes de registros: \numlist{10; 50; 100; 500; 1
  000; 5 000; 10 000; 50 000; 100 000; 500 000; 1 000 000; 5 000 000; 10 000
  000; 50 000 000; 100 000 000; 500 000 000; 1 000 000 000}.

O processo de análise consitirá em executar cada um dos algorimtos implementados
para cada um dos arquivos de entrada e registrar os tempos de execução. Em
seguida, para cada algoritmo, deve-se desenhar o gráfico correspondente aos
dados coletados e analisar se os dados coletados estão de acordo com a
complexidade esperada do algoritmo. Por fim, deve-se fazer uma análise conjunta
de todos os algoritmos de ordenação e determinar para que faixa de tamanho da
entrada cada um dos algoritmos é mais eficiente.

Todos os valores coletados devem ser apresentados tanto na forma de tabelas
quanto na forma de gráficos para facilitar a análise dos dados. Uma recomendação
que pode economizar tempo na confecção dos relatórios é utilizar o comando
\verb|\csvreader| do pacote \texttt{csvsimple} do \LaTeX{}. Os gráficos podem ser
gerados utilizando-se qualquer pacote de software. Uma recomendação é utilizar a
ferramenta \texttt{gnuplot}.


\subsection{Arquivos de Dados}

Os arquivos de dados estarão no formato CSV e serão compostos pelos seguintes
campos:

\begin{enumerate}
\item Email (\texttt{email}); tipo: texto.
\item Sexo (\texttt{gender}); tipo: caractere; valores válidos: \texttt{M},
  \texttt{F}, \texttt{O}.
\item Identificador de Usuário (\texttt{uid}); tipo: alfa-numérico; único.
\item Data de nascimento (\texttt{birthdate}); tipo: data; formato: ISO-8601.
\item Altura, em centímetros (\texttt{height}); tipo: inteiro.
\item Peso, em kilogramas (\texttt{weight}); tipo: numérico.
\end{enumerate}

Os arquivos estão disponíveis para \emph{download} no diretório compartilhado
\url{https://drive.google.com/drive/folders/1eeuM04049iAlYT0Ap_lFCMlMQN9CUr8w?usp=sharing},
no Google Drive.


\section{Entrega}

O trabalho é de execução individual (veja a \autoref{sec:polit-de-colab} para
informações sobre colaboração com colegas), e deve ser entregue até a data
definida no AVA.

Deve ser entregue um arquivo ZIP\footnote{Outros formatos, como RAR e LHA, serão
  desconsiderados.} contendo, ao menos, o seguinte:

\begin{enumerate}
\item Um arquivo chamado \texttt{Readme.md} descrevendo o que está sendo
  entregue, e como compilar (se for o caso) e executar o(s) programas no sistema
  GNU/Linux.
\item O código fonte desenvolvido para o projeto -- preferencialmente organizado
  em um diretório próprio.
\item O código fonte em \LaTeX{} do relatório de análise.
\item O relatório de análise em formato PDF.
\item Arquivos \textbf{extras} de dados (caso tenham sido gerados/usados). Note
  que \textbf{não é necessário} enviar os arquivos de dados que foram fornecidos
  pelo professor.
\end{enumerate}


\section{Política de Colaboração}
\label{sec:polit-de-colab}

Resolver um problema algorítmico é um processo criativo. Quando apresentado a um
novo problema, é sua tarefa ``desmontá-lo'' e alcançar seu próprio entendimento.
Este é um processo meticuloso e demorado. Há muito a ser aprendido com o
processo de pensar em soluções para os problemas de tarefa de casa atribuídos.
Obter ajuda de outro lugar destrói esse processo. No entanto, discutir com os
outros depois de ter passado algum tempo com um problema pode ajudar o processo
e trazer à luz outros aspectos do problema.

Você pode discutir problemas de lição de casa comigo ou com outros alunos da sua
turma, depois de ter pensado o suficiente. Mas quando chega a hora de escrever
sua solução, ela deve ser seu próprio trabalho e deve estar em suas próprias
palavras. Se, depois de trabalhar em um problema, você não conseguir resolvê-lo
satisfatoriamente, poderá obter ajuda de outras pessoas, de livros didáticos ou
da Internet. Se você recebeu ajuda de qualquer outra fonte, é necessário citar
sua fonte no local apropriado da lição de casa, ou seja, anote a URL ou o nome
da pessoa ou o autor e título do texto do qual sua solução foi adquirida.

Depois de obter ajuda de alguma fonte, tente escrever a solução com suas
próprias palavras. Se você discutiu com um colega de classe ou amigo e chegou a
uma solução juntos, então vocês dois devem indicar isso em seus trabalhos de
casa. Se você estiver ajudando alguém ou fornecendo sua solução para alguém,
certifique-se de que ela seja anotada como a fonte da solução. Você também pode
indicar que você ajudou essa pessoa com o problema especificado. Se você não
escrever onde obteve ajuda, isso seria considerado uma fraude (``cola'').

Qualquer evidência de fraude (sem citar a fonte) resultará em severa
penalização de todas as partes envolvidas.

\end{document}

% Local Variables:
% ispell-local-dictionary: "brasileiro"
% TeX-master: t
% TeX-command-extra-options: "-shell-escape"
% End:
